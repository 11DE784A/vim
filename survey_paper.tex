\documentclass[a4paper,11pt]{article}
\author{Ayush Singh}
\title{A Historical Survey Paper on the Development of Cryptography}
\date{\vspace{-5ex}}

\makeatletter
\def\@biblabel#1{}
\makeatother

%\setlength{\parskip}{1ex plus 0.5ex minus 0.2ex}
\begin{document}

\maketitle

\section{Introduction}

The ancient Romans (c. 50 A.D.) and Indians (c. 50 A.D.) are said to have known (and used) simple translational and substitution ciphers: Caesar's cipher is a paticularly well-known example of a substitution cipher. It is said that Caesar used a shift value of three, to encrypt his messages, when communicating with his generals \cite{kahn67}. 
Steganography (hiding the existance of message, in order to keep it confidential) also developed in ancient times. 

The rest of this paper will not deal with classical cryptography, it will  instead focus on the development of cryptography following the publication of \emph{`Communication Theory of Secrecy Systems'} \cite{shannon49}.

Open academic research in cryptography started only in 1970s; marked by the adoption of the open Data Encryption Standard (DES) by the United States National Buraeau of Standards in 1977, and the inception of public-key cryptography in 1976 \cite{dh76b} \cite{merkle78}. Some important techniques and methods are discussed at length in the following sections, including the Diffie-Hellman key exchange system, implementation of a public-key cryptosystem using the RSA cipherand the McEliece cryptosystem.

In addition to encryption and decryption of messages, modern cryptography also concerns itself with authentication of users and digital signatures. The application of public key cryptosystems to obtain a way to verify the identity of users has also been discussed.

Current and future research challenges in cryptography are discussed towards the end of the paper. Legal issues and the implications of development in quantum computing for (public-key) cryptography are explored towards the end of the paper.

\begin{thebibliography}{99}
	\bibitem[Diffie, Hellman 1976b]{dh76b}%
		Diffie W, Hellman M. 1976. New directions in cryptography. IEEE Transactions on Information Theory. 22(6):644-654.
	\bibitem[Kahn 1967]{kahn67}%
		Kahn D. 1967. The Codebreakers -- The Story of Secret Writing. Rev Sub. New York (NY). Scribner.
	\bibitem[Merkle 1978]{merkle78}%
		Merkle RC. 1978. Secure communications over insecure channels. Communications of the ACM. 21(4):294-299.
	\bibitem[Shannon 1949]{shannon49}%
		Shannon CE. 1949. Communication theory of secrecy systems. Bell System Technical Journal. 28(4):656-715.


\end{thebibliography}

\end{document}
